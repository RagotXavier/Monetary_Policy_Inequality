%% LyX 2.3.7 created this file.  For more info, see http://www.lyx.org/.
%% Do not edit unless you really know what you are doing.
\documentclass[english]{article}
\usepackage[T1]{fontenc}
\usepackage[latin9]{inputenc}
\usepackage{geometry}
\geometry{verbose,tmargin=2cm,bmargin=2cm,lmargin=2cm,rmargin=2cm}

\makeatletter
\@ifundefined{date}{}{\date{}}
\makeatother

\usepackage{babel}
\begin{document}
\title{Replicating package\emph{: Should monetary policy care about redistribution?
Optimal monetary and fiscal policy with heterogeneous agents. }}
\author{Fran�ois Le Grand, Ala�s Martin-Baillon, Xavier Ragot }

\maketitle
This set of files replicates the graphs and data of\emph{ Should monetary
policy care about redistribution? Optimal monetary and fiscal policy
with heterogeneous agents. }Be careful of not mixing up these files
with other replicating files and they should be located in an independent
folder. Requires Matlab or Octave / Julia / Dynare. Tested on Matlab
2018b, Dynare 5.4 and Julia release v1.10.0. 
\begin{enumerate}
\item All the .ipynb files are Julia (https://julialang.org/) notebook. 
\item All the .m files can be runned in Matlab or Octave (https://octave.org/).
Most require Dynare (https://www.dynare.org/). 
\end{enumerate}

\section{Quantitative assessment of the sticky-price model.}

\subsection{How to run?}

In this order:
\begin{enumerate}
\item Open `Sticky\_Prices/steady\_state/Main\_SP.ipynb' and execute all
cells. 
\item Run `Sticky\_Prices/dynamics/main\_SP.m'.
\item Run `Sticky\_Prices/dynamics/main\_SP.m'. \textbf{Comment the line
3 ``calib = 'baseline' `` and uncomment the line 4 ``calib = refined''}.
\item Run `Sticky\_Prices/dynamics/main\_taylor\_SP.m'.
\item Run `Sticky\_Prices/dynamics/unequal.m'.
\end{enumerate}
Output: 
\begin{enumerate}
\item Figure 1: Run `Sticky\_Prices/dynamics/Do\_IRFs\_SP\_baseline.m'.
\item Tables 7/8: Run `Sticky\_Prices/steady\_state/Tables\_SP.ipynb'.
\item Figure 7: Run `Sticky\_Prices/dynamics/Do\_IRFs\_SP\_unequal.m'.
\end{enumerate}

\subsection{The details}
\begin{itemize}
\item The `Julia` files takes care of computing the steady state, while
the .m file simulates the model in the presence of aggregate shocks. 
\end{itemize}
The output of the `Julia' files is: 
\begin{enumerate}
\item a file `todynare\_SP\_baseline.mat'
\item a file `todynare\_SP\_refined.mat' 
\item a file `To\_IRFs\_SP\_unequal.mat' 
\end{enumerate}
Those files will be used by main.m, main\_taylor.m, unequal.m
\begin{itemize}
\item The outputs of the `Octave' / `Matlab' files are: 
\begin{itemize}
\item `To\_IRFs\_SP\_baseline.mat' 
\item `To\_IRFs\_SP\_taylor.mat'
\item `To\_IRFs\_SP\_unequal.mat'
\end{itemize}
\end{itemize}
Then:
\begin{itemize}
\item `Do\_IRFs\_SP\_baseline.m' generate the Figure 1: `IRFs\_SP\_Eco\_1\_2\_taylor.png' 
\item `Do\_IRFs\_SP\_unequal.m' generate the Figure 7: `IRFs\_SP\_uneq\_Eco\_1\_2.png' 
\end{itemize}
.The output of the `Julia' file is also:
\begin{enumerate}
\item 6 files: moments\_eco1/2\_baseline, 'moments\_taylor', moments\_eco1/2\_refined
\end{enumerate}
Those files will be used by the Julia file `Tables.ipynb' to display
the tables for First- and second-order moments for key variables. 

\subsection{The steady state computation}

The steady state is computed thanks to nine `Julia` notebooks. Each
of above files are commented and self-explained.
\begin{itemize}
\item `Main.ipynb': Solves the steady-state model and returns the truncated
model (as `steady\_state\_dynare.mat` for `Dynare`, saved in the current
folder);
\item `Structures.ipynb`: Structures and parameter calibration from targets;
\item `Utils.ipynb`: Contains some useful functions;
\item `SolveAiyagari.ipynb`: Solves the Aiyagari model;
\item `Projection.ipynb`: Computes the steady-state truncated model;
\item `Projection\_ref.ipynb`: Computes the steady-state redined truncated
model;
\item `Ramsey.ipynb`: Computes the steady-state Lagrange multipliers.
\item `Simulation.ipynb': : Contains a function used to display tables of
first and second order moments of key variables;
\item `Tables.ipynb': Display tables of first and second order moments of
key variables;
\end{itemize}
Simulating the model with aggregate shocks, 
\begin{itemize}
\item The file `main.m` simulates the model for the first two economies
of the paper. For the baseline truncation or the refined truncation
depending on the line 3 and 4 of the file. 
\begin{itemize}
\item Economy 1: optimal inflation
\item Economy 2: constant inflation
\end{itemize}
\item The file `main\_taylor.m` simulates the model for third economy of
the paper:
\begin{itemize}
\item Economy 3: Taylor Rule
\end{itemize}
\end{itemize}
The outcomes of the program can be parametrized as follows:
\begin{itemize}
\item The `Octave' / `Matlab' actually writes different `Dynare' codes 
\begin{itemize}
\item `code\_dynare\_baseline1.mod', `code\_dynare\_baseline2.mod', `code\_dynare\_taylor.mod'
which correspond to the three economies of the baseline specifiation. 
\item `code\_dynare\_refined1.mod', `code\_dynare\_refined2.mod'.
\item `code\_dynare\_unequal1.mod', `code\_dynare\_unequal2.mod'
\end{itemize}
\end{itemize}
Each of this code is then solved in `Dynare'. These files, as interim
Dynare files are created in the current folder.

\subsection{Comparisons with the Reiter method}

\subsubsection{How to run?}

In this order:
\begin{enumerate}
\item Open `Reiter/steady\_state/Main\_Reiter\_Comp.ipynb' and execute all
cells.
\item Run `Reiter/dynamics/reiter\_comp.m'. 
\item Run `Reiter/dynamics/refined\_comp.m'. 
\item Run `Reiter/dynamics/code\_difference.m' . 
\end{enumerate}
Output: 
\begin{itemize}
\item Figure 8: run `Reiter/dynamics/fig\_Comp\_Reiter\_Trunc.m' 
\item Table 11: run `Reiter/steady\_state/Tables\_Reiter\_Comp.ipynb' cells
1 to 4 
\item Table 12 : run `Reiter/steady\_state/Tables\_Reiter\_Comp.ipynb' cell
5
\end{itemize}

\subsubsection{The details}

The output of the `Julia' files is: 
\begin{itemize}
\item a file `todynare\_Comp\_Reiter.mat'
\item a file `todynare\_Comp\_Refined.mat'
\end{itemize}
Those files will be used by `reiter\_comp.m, `refined\_comp.m' 
\begin{itemize}
\item The outputs of the `Octave' / `Matlab' files are: 
\begin{itemize}
\item `todiff\_Reiter IRFs.mat' save the IRFs of the Reiter simulation 
\item `todiff\_Comp\_Refined.mat' save the IRFs of the refined truncation
\item `moments\_Comp\_Reiter.mat' save the moments of the Reiter simulation 
\item `moments\_Comp\_Refined.mat' save the moments of model with the refined
truncation 
\item `to\_code\_difference\_reiter.mat' save the result of the Reiter simulation 
\item `to\_code\_difference\_refined.mat' save the result of the refined
truncation simulation
\item Then those files will be used by `code\_difference.m' to generate
`diff\_Reiter.mat': save the result for table 12. 
\end{itemize}
\item Finally: 
\begin{itemize}
\item `Reiter/steady\_dynamics/fig\_Comp\_Reiter\_Trunc.m' generates the
Figure 8: `Comp\_Reiter\_Trunc.png' 
\item `Reiter/steady\_state/Tables\_Reiter\_Comp.m' displays Table 11 and
12. 
\end{itemize}
\end{itemize}

\section{Quantitative assessment of the sticky-wage model.}

\subsection{How to run?}

In this order:
\begin{enumerate}
\item Open `Sticky\_Wages/steady\_state/Main\_SW.ipynb' and execute all
cells. 
\item Run `Sticky\_Wages/dynamics/main\_SW.m'. 
\item Run `Sticky\_Wages/dynamics/main\_taylor\_SW.m'.
\end{enumerate}
Output 
\begin{enumerate}
\item Graph : Run `Sticky\_Wages/dynamics/Do\_IRFs\_SW\_baseline.m' 
\end{enumerate}

\subsection{The details}
\begin{itemize}
\item The `Julia` files takes care of computing the steady state, while
the .m file simulates the model in the presence of aggregate shocks.
\begin{itemize}
\item The output of the `Julia` file is a file `todynare\_SW.mat' that will
be used by `Octave` / `Matlab`. 
\end{itemize}
\item The outputs of the .m are: 
\begin{itemize}
\item `To\_IRFs\_SW.mat' 
\item `To\_IRFs\_SP\_taylor.mat'
\end{itemize}
\item `Do\_IRFs\_SP\_baseline.m' generate IRFs\_SW\_Eco\_1\_4\_taylor.png 
\end{itemize}

\subsection{The steady state computation}

The steady state is computed thanks to seven `Julia` notebooks. Each
of above files are commented and self-explained.
\begin{itemize}
\item `Main.ipynb': Solves the steady-state model and returns the truncated
model (as `steady\_state\_dynare.mat` for `Dynare`, saved in the current
folder);
\item `Structures.ipynb': Structures and parameter calibration from targets;
\item `Utils.ipynb': Contains some useful functions;
\item `SolveAiyagari.ipynb': Solves the Aiyagari model;
\item `Projection.ipynb': Computes the steady-state truncated model;
\item `Ramsey\_SP.ipynb': Computes the steady-state Lagrange multipliers.
\item `ToDynare\_SW': Generate the output of the Julia file: `todynare\_SW.mat'
that will be used by `Octave` / `Matlab`. 
\end{itemize}
Simulating the model with aggregate shocks, 
\begin{itemize}
\item The file `main\_SW.m` simulates the model for the first two economies
of the section:
\begin{itemize}
\item Economy 1: optimal inflation
\item Economy 2: constant wage inflation
\end{itemize}
\item The file `main\_SW\_taylor.m` simulates the model for third economy
of the paper:
\begin{itemize}
\item Economy 3: Taylor Rule
\end{itemize}
\end{itemize}
The outcomes of the program can be parametrized as follows:
\begin{itemize}
\item The `Octave` / `Matlab` actually writes three `Dynare' codes `code\_dynare\_SW1.mod',
`code\_dynare\_SW4.mod', `code\_dynare\_SW\_taylor.mod' which correspond
to the three economies. Each of this code is then solved in `Dynare'.
These files, as interim Dynare files are created in the current folder.
\end{itemize}

\end{document}
